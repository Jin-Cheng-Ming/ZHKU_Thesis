%---------------------------------------------------------------------------%
%->> 文末事项 Backmatter
%---------------------------------------------------------------------------%

%-> 致谢 Acknowledge
% 语法 syntax: \chapter[目录]{标题}\chaptermark{页眉}
{
    \ctexset {
        chapter = {
            format = \linespread{1.0}\heiti\zihao{-3}\sffamily\centering,
            pagestyle = noheaderstyle,
            beforeskip = {0pt plus 0pt minus 0pt},
            afterskip = {0pt plus 0pt minus 0pt},
        }
    }
    \chapter[\acknowledgeNameMark]{\acknowledgeName}
}

此处填写致谢。

\lipsum[1-3]

\rightline{2024年6月}

\cleardoublepage[plain]

%-> 简介与成就 Resume and achievement
{
    \ctexset {
        chapter = {
            format = \linespread{1.0}\heiti\zihao{-3}\sffamily\centering,
            pagestyle = noheaderstyle,
        }
    }
    \chapter[\achievementNameMark]{\achievementName}
}

\section*{作者简历:}
××××年××月——××××年××月,在××大学××院(系)获得学士学位。

××××年××月——××××年××月,在××大学××院(系)获得硕士学位。

××××年××月——××××年××月,在中国科学院××研究所(或中国科学院大学××院系)攻读博士/硕士学位。

\section*{工作经历:}

\lipsum[1-2]

\section*{已发表(或正式接受)的学术论文:}

{
    % restore default behavior
    \setlist[enumerate]{}
    \begin{enumerate}
        \item 已发表的工作1
        \item 已发表的工作2
    \end{enumerate}
}

\section*{申请或已获得的专利:}

(无专利时此项不必列出)

\section*{参加的研究项目及获奖情况:}

\lipsum[1]

\cleardoublepage[plain]% 让文档总是结束于偶数页,可根据需要设定页眉页脚样式,如 [noheaderstyle]
%---------------------------------------------------------------------------%
