\chapter[研究生学位论文撰写规范]{仲恺农业工程学院研究生学位论文撰写规范\\(2023年12月修订)}\footnotetext{来源:关于印发《仲恺农业工程学院研究生学位论文写作规范(2023年修订)》的通知 https://yjs.zhku.edu.cn/info/1060/4945.htm}
{
\let\cleardoublepage\relax
}

为进一步规范研究生学位论文撰写,提高学位论文质量,根据国家相关标准和我校实际情况,特制定如下规范:

\section{研究生学位论文结构}

学位论文基本结构包括前置部分、主体部分和附录部分。

\subsection{前置部分}

\begin{enumerate}
    \item 封面
    \item 原创性声明、版权使用授权书、学位论文提交同意书
    \item 中文摘要、英文摘要
    \item 英文缩略词或符号表
    \item 目录
\end{enumerate}

\subsection{主体部分}

\begin{enumerate}
    \item 正文
    \item 参考文献
\end{enumerate}

\subsection{附录部分}

\begin{enumerate}
    \item 不便列入正文的材料、数据等
    \item 攻读学位期间的研究成果、获奖情况和参研项目等
    \item 致谢
\end{enumerate}

\section{研究生学位论文写作要求}

\subsection{封面及书脊}

\begin{enumerate}
  \item 封面内容包括学校代码、分类号、密级、论文题目、学号、作者姓名、指导教师、学院、专业(领域)等。
  \item 分类号:注明《中国图书资料分类法》的类号。
  \item 密级:必须按国家规定的保密条例和学位论文开题前申请的密级填写,非涉密论文不能填写。涉密学位论文须在封面“密级”后标明密级和保密期限,标志符号为“★”,“★”前标密级,“★”后标保密期限,例如“秘密★5年”。
  \item 题目:学位论文题目要精炼,起画龙点睛的效果,字数一般不得超过26个字。
  \item 作者姓名:填写研究生姓名。
  \item 指导教师:所列出的指导教师必须是已在研究生部备案的指导教师。
  \item 学院:填申请学位所在学院。
  \item 专业(领域)名称:按国家颁布的学科、专业目录中的名称填写。
  \item 书脊内容包括学位论文题目、作者姓名、学校名称和年份。
\end{enumerate}

\subsection{原创性声明、版权使用授权书、学位论文提交同意书}

\begin{enumerate}
  \item 原创性声明:论文作者签名。
  \item 版权使用授权书:论文作者和导师签名。
  \item 学位论文提交同意书:导师审核学位论文并签名。
\end{enumerate}

\subsection{中文摘要}

摘要是论文的缩影,语言力求精炼准确。应概括论文的主要信息,包括研究目的、内容、结果和结论,要重点突出论文的新见解或创新性。硕士论文摘要字数800字左右。

关键词采用能覆盖论文主要内容的通用技术词条作为关键词,一般3至5个,按涉及的内容、领域从大到小排在摘要下方。

\subsection{英文摘要}

英文摘要内容与中文摘要基本一致,写作力求符合科技英语文法要求,英文摘要由论文题目、作者、作者单位、正文、关键词组成。

\subsection{英文缩略词或符号表}

如有英文缩略词或符号表,应放在英文摘要之后,目录之前。

\subsection{目录}

目录置于摘要之后。目录正文需列出论文各章节标题及页码。目录中各层次标题与正文层次标题一致。原创性声明、版权使用授权书、学位论文提交同意书、中文摘要、英文摘要不编入目录中。致谢、参考文献、附录一律不编序号。

\subsection{正文}

正文是学位论文的主体部分,论文内容必须立论正确、言之成理、论据可靠、阐述透彻、逻辑严密,论文书写要层次分明、思路清晰、文字简练、结构完整。正文撰写必须严格遵守学术规范,论文中如引用他人的论点或数据资料,必须注明出处,引用合作者的观点或研究成果时,要加注说明,否则将被视为剽窃行为。正文部分具体研究内容应是论文作者自己的研究成果,不能将他人研究成果不加区分地掺和进来。结论中要严格区分自己取得的成果与导师及他人的科研工作成果,评价研究工作成果时,要实事求是。

学位论文的研究主题切忌过大,通常只有一个主题(不能是几部分研究工作的简单拼凑),该主题应针对某学科领域中的一个具体问题展开深入、系统地研究,并得出有价值的研究结论。

\begin{enumerate}

    \item 正文结构
    
    由于学位论文研究工作涉及的学科专业特点、论文选题情况研究方法、工作进程、结果表达方式、学位论文形式等有很大的差异,学位论文正文结构有不同的写作方式。正文的撰写结构具体参考《研究生学位论文正文结构规范》。

    \item 各级标题
    
    正文层次标题应简短明确,以不超过15字为宜,题末不加标点符号。各层次一律用阿拉伯数字连续编号,(如“1”,“2.1”,“3.1.2”)一律左对齐,后空一个字符写标题。各级标题与段落之间不留空行。
    
    \item 图、表、公式
    
    (1)图、表力求简明,自明性强。图与表的内容不得重复,并尽可能紧随相应的文字描述排列。图、表一般不得分页排列,若无法避免,须在另一页制作独立表头。
    
    (2)表一般应该采用“三线式”,表格的上下边框线宽度为1.5磅,中线宽度为0.5磅。有分图时,分图用a、b、c表示。
    
    (3)图、表与正文之间须设置段前段后0.5行间距,图题、表题与图、表之间不留空行。
    
    (4)学位论文中的图题、表题应采用中英文对照,中文居上。
    
    (5)图题居图下方,表题居表上方,用阿拉伯数字编号,如:图1或图1.1(表1或表1.1),图号后空1个字符写图题。
    
    (6)论文若有多个公式,按章节采用(1.1)、(1.2)等编号方式,公式编号写在右边行末,其间不加虚线。
    
    \item 数据分析
    
    论文中试验数据的统计分析,如果是应用计算机软件的,尽可能采用公开发行的程序;如果是自编的,应在附录中列出程序。在数据分析中各试验数据的平均数之后应列出平均数的标准误(S.E.)。如列出标准误(S.D.)的要注明样本数n值。
    
    \item 注释
    
    注释是论文中的解释性说明词句,采用“脚注”方式,以右上标的形式在注释处标注序号(圆括号加数字),并在当前页下按序号顺序列出注释的内容。注释的序号每页单独排序。
    
    \item 量和单位
    
    应严格执行GB 3100-1993、GB/T 3101-1993、GB/T 3102.1~13-1993有关量和单位的规定(参阅《常用量和单位》.计量出版社,1996)。单位名称的书写,可采用国际通用符号,也可用中文名称,但全文应统一,不要两种混用。
    (1)文中所用的量度单位按“中国高等学校自然科学学报编排规范”(北京工业大学出版社,1993)中“附录B量和单位”的规定,如公斤用kg。但在正文叙述时,应用中文表述,如:“每克”,而不要用“每g”。
    (2)文中采用英文字母缩写的,第一次出现时应把英文的全称写出,如:Gross National Product(GNP)、Diamond Back Moth(DBM)。

\end{enumerate}

\subsection{参考文献}

参考文献格式基本遵守GB/T7714-2015《信息与文献——参考文献著录规则》,但主要有以下两点不同:

\begin{enumerate}
  \item 正文中的引文标注
  
  默认形式是“(作者,出版年)”。当文中已提到作者时,只需在括号中注明出版年,即“作者(出版年)”。若在同一处引用多篇参考文献,则改为“(作者,出版年;作者,出版年)”并按出版年份先后排序。
  
  \item 文后参考文献列表
  
  建议采用Note Express软件进行处理,先选用“中华人民共和国国家标准\_GBT\_7714-2015”样式,随后须进行手动修改。
  
  引文开头的序号、空格应删除;取消段落缩进,全选并点击右键,在“段落”选项内将“悬挂缩进”改为“无”。列表按作者姓名排序,中文文献在前,外文文献在后。(详见《学位论文中写作中使用Note Express处理文献引用指南》)
  
\end{enumerate}

\subsection{附录}

如有附录,应编入目录中。附录是正文主体的补充。攻读学位期间发表的(含已录用,并有录用通知书的)与学位论文相关的学术论文,由于篇幅过大或取材于复制件不便编入正文的材料、数据,对本专业同行有参考价值但对一般读者不必阅读的材料,论文所使用计算机程序清单、软磁盘,成果鉴定证书、获奖奖状或专利证书的复印件等可作为附录内容。有多项附录内容时,采用附录A、附录B等编号。

\subsection{致谢}

致谢中的用词和用语不要过于渲染。内容应简洁明了、实事求是。凡在读研究生不得称谓硕士。

\subsection{其他}

论文中的物理量名称、符号及计量单位一律采用国务院发布的《中华人民共和国法定计量单位》,单位名称和符号的书写方式,应采用国际通用符号。

除特殊需要外,全文用汉语简化字,英文数字用“Times New Roman”字体。

资助论文的科研项目可在“致谢”中标出。

学位论文要求用中文写作。外国留学生或与国际研究课题合作完成的论文可用英文撰写,但必须用中文撰写较详细的“摘要”。


\section{研究生学位论文格式要求}

\subsection{论文纸张要求}

采用A4纸(21cm×29.7cm)双面打印。封面纸质为180-200g/m²,硕士学位论文封面为橙色,其他部分用普通白色A4纸。

\subsection{页面设置}

\begin{enumerate}
  \item 边距:上、下边距2.5cm,左、右边距2.7cm,页眉1.5cm,页脚1.75cm。
  \item 行间距:1.5倍。
  \item 页码:页码居中。前置部分从中文摘要到目录的页码用大写罗马数字,主体部分用阿拉伯数字。
\end{enumerate}

\subsection{字体、字号与排版要求}

% FIXME 表格里面列表结束之后会有空一行
\begin{center}
  \begin{longtable}{m{4cm}m{10cm}}
    \bicaption{\quad 字体、字号与排版要求}{\quad Font, Size, Typography Requirements} \label{tab:printrequirements} \\
  
    \hline\hline 
    \textbf{内容} & \textbf{格式要求} \\ 
    \hline\hline
    \endfirsthead
    
    \multicolumn{2}{l}{{续上表}}\\
    \hline\hline 
    \multicolumn{1}{c}{\textbf{内容}} & \multicolumn{1}{c}{\textbf{格式要求}} \\ 
    \hline\hline
    \endhead
    
    \hline 
    \multicolumn{2}{r}{{接下表}} \\
    \endfoot
    
    \hline\hline
    \endlastfoot
    
    封面及书脊 & 要求按照示例模板。\\
    原创性声明、版权使用授权书、论文提交同意书 & 要求按照示例模板。\\
    中文摘要及关键词 & 
    \begin{enumerate}
      \item “摘要”两字居中,两字之间留空4个字符,四号黑体,摘要正文小四号宋体。
      \item “关键词”三个字小四号黑体,与摘要靠左对齐,后加“:”,关键词之间用分号隔开,小四号宋体。
    \end{enumerate}\\
    英文摘要及关键词 & 
    \begin{enumerate}
      \item 英文题目小三号 “Times New Roman”字体、加粗、居中。
      \item “Abstract”及“Key words:”靠左对齐,小四号“Times New Roman”字体、加粗。
      \item 英文摘要正文和英文关键词为小四号“Times New Roman”字体。          
      \item 英文题目和关键词,每一个实词的第一个字母大写,关键词之间用分号隔开。
    \end{enumerate}\\
    英文缩略词或符号表 & 
    \begin{enumerate}
      \item 标题“英文缩略词或符号表”用小三号黑体,居中;
      \item 缩略词或符号表中的中文用小四号宋体,英文用小四号“Times New Roman”字体,靠左对齐。
    \end{enumerate}\\
    目录 & 
    \begin{enumerate}
      \item “目录”两字居中,两字之间留空4个字符,小三号黑体。
      \item 目录中各层次标题与正文层次标题一致,一律用阿拉伯数字排序,不同层次的数字之间用圆点相隔,一般不超过3级层次。        
      \item 目录正文用小四号宋体,层次标题序号一律左对齐,页码右对齐,中间用小黑点连接。
    \end{enumerate}\\
    正文 & 
    \begin{enumerate}
      \item 一级标题小三号黑体、二级标题四号黑体、三级标题小四号黑体,正文除标题外
      的其他部分小四号宋体(中文)或小四号“Times New Roman”字体(英文)。
      \item 各层次一律用阿拉伯数字连续编号,如:“1”,“2.1”,“3.1.2”,一律左对齐,后空一个字符写标题。各级标题与段落之间不留空行。       
      \item 正文中的注释用五号楷体。
      \item 图题、表题使用加粗小四号宋体。
    \end{enumerate}\\
    参考文献 & “参考文献”四字居中,小三号黑体。参考文献的正文,中文用五号宋体、英文及阿拉伯数字用五号“Times New Roman”字体。\\
    附录 & “附录”及相应标题内容用小三号黑体,正文用小四号宋体,英文用相应字号的“Times New Roman”字体。\\
    致谢 & 
    \begin{enumerate}
      \item “致谢”两字居中,两字之间留空4个字符,小三号黑体。        
      \item 致谢的正文小四号宋体。       
    \end{enumerate}\\
    其他 & 
    \begin{enumerate}
      \item 全文中阿拉伯数字和英文均使用“Times New Roman”字体,字号与相应部分内容的汉字一致。
      \item 拉丁学名采用右斜体字母。
    \end{enumerate}\\
  \end{longtable}
\end{center}