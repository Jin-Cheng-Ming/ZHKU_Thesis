\chapter[研究生学位论文正文结构规范]{研究生学位论文正文结构规范\protect\footnote{来源:关于印发《仲恺农业工程学院研究生学位论文写作规范(2023年修订)》的通知\\ https://yjs.zhku.edu.cn/info/1060/4945.htm}}
% \thispagestyle{noheaderstyle}

  正文是学位论文的主体部分。为进一步规范我校研究生学位论文正文部分的撰写,特制定本规范。

  \section{正文部分的写作要求}
  \begin{enumerate}
    \item 
    
    由于学位论文研究工作涉及的学科专业特点、论文选题方向、研究方法手段、工作进程进展、结果表达方式、论文形式要求等有很大的差异,学位论文正文部分有不同的写作方式。但论文内容必须立论正确、言之成理、论据可靠、阐述透彻、逻辑严密,写作必须严格遵循本学科国际通行的学术规范,书写层次分明、思路清晰、文字简练、结构完整。
    
    \item 
    
    学位论文应在导师指导下独立完成,其选题必须属于申请学位的学科、专业范畴。研究生应结合研究内容、学科专业、学位类型合理安排学位论文的正文结构。
    
    \item 
    
    专业学位论文形式及正文结构,须同时参考各专位研究生教育指导委员会的相关规定和要求。
  \end{enumerate}

  \section{正文结构基本要求}

    \begin{enumerate}
        \item 硕士学位论文原则上不分章撰写,按1/1.1/1.1.1三级标题编排。
        \item 学位论文正文部分一般分三级标题(最多四级标题),层次应清楚,标题应简明扼要,重点突出。排版格式需符合《仲恺农业工程学院研究生位论文撰写规范》。
        \item 研究生学位论文的正文部分,一般包括前言、论述主体、结论等内容。
        
          \begin{enumerate}
            \item 前言
            
            主要阐述论文的研究背景与选题意义、国内外研究现状、本论文要解决的关键科学问题、论文运用的主要理论与方法、基本研究思路、技术路线与论文的整体结构安排等。其中国内外研究现状(文献综述)部分,要对本研究主题范围内的文献进行提炼与升华,近五年文献占比不低于1/3,既要有综合叙述,也要提出自己的看法和观点。着重总结论文的创新点或新见解。

            人文社科类学位论文导论部分可根据论文需要,适当调整有关内容,亦可增加有关“相关概念”、“理论基础”等内容。

            \item {论述主体}
            
            具体内容可因研究课题的性质、学科专业的类别而有所不同,可以包括调查对象、实验和观测方法、仪器设备装置和测试方法、材料原料、实验和观测结果、计算方法和编程原理、模型或方案设计、经过加工整理的图表、对实验结果的分析、形成的论点与理论分析、导出的结论等,要求实事求是、重点突出、逻辑清晰、层次分明。

            人文社科类学位论文论述主体部分可根据论文需要,适当调整有关内容,如调查方法、样本与数据来源、研究设计、模型构建或统计方法、案例论证或实证分析、对调研结果/模型运行结果的分析等。
            
            \item {结论}

            主要是对论文主要研究成果、论点的提炼与概括,阐述研究工作的创造性及所取得的研究成果在本学术领域中的地位及理论价值与应用价值,并指出今后在本研究方向进行深入研究工作的展望与设想,表述应明确、简炼、完整、准确、有条理,应严格区分本人的研究成果与导师或其他人科研成果。

            人文社科类学位论文可在该部分提出政策建议或者政策涵义或政策启示的分析。政策建议一般是从政府的角度提出与研究结论相应的、具有针对性的有利于解决问题、推动经济发展、社会进步的政策措施。

          \end{enumerate}

    \end{enumerate}

  \section{正文结构样式(仅供参考)}

  以下几类正文结构样式供参考。各学院、一级学科可根据实际情况制定本学科的正文结构样式(特别提示:仅限于学位论文的正文结构,其他写作规范、格式要求必须按照学校有关统一规定)。

  % FIXME 灰色背景文本块不能跨页
  \begin{enumerate}
      \item 版式一(自然科学类硕士学位论文参考使用)
      \begin{tcolorbox}[breakable]{
        1 前言\\
        2 材料与方法\\
        2.1\\
        2.2\\
        ……\\
        3 结果与分析\\
        3.1\\
        3.2\\
        ……\\
        4 讨论\\
        4.1\\
        4.2\\
        ……\\
        5 结论\\
        5.1\\
        5.2\\
        ……
      }\end{tcolorbox}

      \item 版式二(人文社科类硕士学位论文参考使用)
      \begin{tcolorbox}[breakable]{
        1 前言\\
        2 研究设计(或调研方法、案例选择与分析方法等)\\
        2.1\\
        2.2\\
        ……\\
        3 实证分析(或调研资料与数据分析、案例资料与数据分析等)\\
        3.1\\
        3.2\\
        ……\\
        4 讨论与结论(或结论与展望、政策与对策建议等)\\
        4.1\\
        4.2\\
        ……
      }\end{tcolorbox}
      
      \item 版式三(艺术类硕士学位论文参考使用)
      \begin{tcolorbox}[breakable]{
        1 前言\\
        2 研究设计(或调研方法、设计理论等)\\
        2.1\\
        2.2\\
        ……\\
        3 设计分析(或调研资料与数据分析、案例资料与数据分析等)\\
        3.1\\
        3.2\\
        ……\\
        4 设计实践\\
        4.1\\
        4.2\\
        ……\\
        5 结论
      }\end{tcolorbox}
  \end{enumerate}

    
