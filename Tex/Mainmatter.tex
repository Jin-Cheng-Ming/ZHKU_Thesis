%---------------------------------------------------------------------------%
%->> Main content
%---------------------------------------------------------------------------%
\chapter{LaTeX使用说明}\label{chap:guide}
% 开始页码
\setcounter{page}{1}
\thispagestyle{mainmatterstyle}

{
为方便使用及更好地展示LaTeX排版的优秀特性,ucasthesis的框架和文件体系进行了细致地处理,尽可能地对各个功能和板块进行了模块化和封装。

\section{初步设置}

\begin{enumerate}
    \item 使用overleaf:打开并注册\href{https://cn.overleaf.com/}{overleaf}。
    \item 将整个文件夹上传至overleaf项目。
    \item 右键菜单,设置编译器为XeLaTeX,选择TexLive 2021
    \item 点击编译,即可预览PDF文件
\end{enumerate}

编译完成即可获得本PDF说明文档。

\section{文档目录简介}

\subsection{Thesis.tex}

Thesis.tex为主文档,其设计和规划了论文的整体框架,通过对其的阅读可以了解整个论文框架的搭建。

\subsection{编译脚本}

为方便本地编译,提供bat脚本和.sh脚本分别用于windows环境和unix环境。

\begin{itemize}
    \item Windows:双击Dos脚本artratex.bat可得全编译后的PDF文档,其存在是为了帮助不了解LaTeX编译过程的初学者跨过编译这第一道坎,请勿通过邮件传播和接收此脚本,以防范Dos脚本的潜在风险。
    \item Linux或MacOS:在terminal中运行
        \begin{itemize}
            \item \verb|./artratex.sh xa|:获得全编译后的PDF文档
            \item \verb|./artratex.sh x|:快速编译,不会生成文献引用
        \end{itemize}
\end{itemize}

全编译指运行 \verb|xeLaTeX+bibtex+xeLaTeX+xeLaTeX| 以正确生成所有的引用链接,如目录、参考文献及引用等。在写作过程中若无添加新的引用,则可用快速编译,即只运行一遍LaTeX编译引擎以减少编译时间。

\subsection{Tmp文件夹}

运行编译脚本后,编译所生成的文档皆存于Tmp文件夹内,包括编译得到的PDF文档,其存在是为了保持工作空间的整洁。

\subsection{Style文件夹}

包含ucasthesis文档类的定义文件和配置文件,通过对它们的修改可以实现特定的模版设定。

\begin{enumerate}
    \item ucasthesis.cls:文档类定义文件,论文的最核心的格式即通过它来定义的。
    \item ucasthesis.cfg:文档类配置文件,设定如目录显示为“目~录”而非“目录”。
    \item artratex.sty: 常用宏包及文档设定,如参考文献样式、文献引用样式、页眉页脚设定等。这些功能具有开关选项,常只需在Thesis.tex中进行启用即可,一般无需修改artratex.sty本身。
    \item artracom.sty:自定义命令以及添加宏包的推荐放置位置。
\end{enumerate}

\subsection{Tex文件夹}

文件夹内为论文的所有实体内容,正常情况下,这也是使用ucasthesis撰写学位论文时,主要关注和修改的一个位置,注:所有文件都必须采用UTF-8编码,否则编译后将出现乱码文本,详细分类介绍如下:

\begin{itemize}
    \item Frontinfo.tex:为论文中英文封面信息。论文封面会根据英文学位名称如Master,Doctor自动切换为相应的格式。
    \item Frontmatter.tex:为论文前言内容如中英文摘要等。
    \item Mainmatter.tex:索引需要出现的Chapter。开始写论文时,可以只索引当前章节,以快速编译查看,当论文完成后,再对所有章节进行索引即可。
    \item Chap{\_}xxx.tex:为论文主体的各章,可根据需要添加和撰写。添加新章时,可拷贝一个已有的章文件再重命名,以继承文档的 UTF8 编码。
    \item Appendix.tex:为附录内容。
    \item Backmatter.tex:为发表文章信息和致谢部分等。
\end{itemize}

\subsection{Img文件夹}

用于放置论文中所需要的图类文件,支持格式有:.jpg, .png, .pdf。其中,\verb|ucas_logo.pdf|为国科大校徽。

\subsection{Biblio文件夹}

 ref.bib用于放置论文中所需要参考文献信息。

\section{功能介绍}

\subsection{参考文献引用}

参考文献引用过程以实例进行介绍,假设需要引用名为"Document Preparation System"的文献,步骤如下:

1)将Bib格式的参考文献信息添加到ref.bib文件中(此文件位于Biblio文件夹下),如直接粘贴自网站,请注意修改其格式。

2)索引第一行 \verb|@article{lamport1986document,|中 \verb|lamport1986document| 即为此文献的label (中文文献也必须使用英文label,一般遵照:姓氏拼音+年份+标题第一字拼音的格式),想要在论文中索引此文献,\verb|\citep{lamport1986document}|。如此处所示 \textsuperscript{\citep{lamport1986document}}。

多文献索引用英文逗号隔开, 如此处所示 \textsuperscript{\citep{lamport1986document, chu2004tushu, chen2005zhulu}}。

更多例子如:

Walls等\textsuperscript{\citep{walls2013drought}}根据Betts\textsuperscript{\citep{betts2005aging}} 的研究,首次提出......理论。其中关于......的研究\textsuperscript{\citepns{walls2013drought, betts2005aging}},是当前中国得到迅速发展的研究领域 \textsuperscript{\citep{chen1980zhongguo, bravo1990comparative}}。

不同文献样式和引用样式,如著者-出版年制(authoryear)、顺序编码制(numbers)、上标顺序编码制(super)可在Thesis.tex中对artratex.sty调用实现,详见 \href{https://github.com/mohuangrui/ucasthesis/wiki}{ucasthesis 知识小站之文献样式}。

%若在上标顺序编码制(super)模式下,希望在特定位置将上标改为嵌入式标,可使用 \citetns{niu2013zonghe,stamerjohanns2009mathml} 和 \citepns{niu2013zonghe,stamerjohanns2009mathml}。

参考文献索引的更多知识,请见 \href{https://en.wikibooks.org/wiki/LaTeX/Bibliography_Management}{WiKibook Bibliography}。\nocite{*}% 使文献列表显示所有参考文献(包括未引用文献)


}
\cleardoublepage[noheaderstyle]
%---------------------------------------------------------------------------%
