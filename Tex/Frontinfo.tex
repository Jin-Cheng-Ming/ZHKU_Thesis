%---------------------------------------------------------------------------%
%->> 封面信息 Titlepage information
%---------------------------------------------------------------------------%

%-
%-> 中文封面信息
%-

%- 通用信息
% 学校编码:一般不变,保持即可
\schoolCode{11347}
% 学号:按照自己的学号填写
\studentCode{2112907013}
% 按照实际情况填写
\classificationCode{S476}	
% 密级:按照实际情况填写,公开则不填
\confidential{} 
% 校徽与校名:一般不变,保持即可
\schoolLogo[height=2.2cm]{zhku_logo}
%- 标题:第一个空位中括号(可选,默认封面标题)给书签文件名等预留的(一般一行字),第二个空位大括号(必填)是给封面标题排版的(可自定义换行),按实际情况填写
% 中文标题
% \title{东洋区冠果蝇亚科系统分类学研究}
\title[东洋区冠果蝇亚科系统分类学研究(双翅目:果蝇科)]{东洋区冠果蝇亚科系统分类学研究\\(双翅目:果蝇科)}
% 英文标题
% \enTitle{Taxonomic Studies on the Subfamily Drosophila(Diptera: Drosophilidae)}
\enTitle[Taxonomic Studies on the Subfamily Drosophila(Diptera: Drosophilidae)]{Taxonomic Studies on the Subfamily Drosophila\\(Diptera: Drosophilidae)}
% 作者
\author{高~~某~~某}

% 培养单位:自己所在的学院或者研究所等,按照实际情况填写
\institute{农业与生物学院}
% 专业:按照实际情况填写
\major{农业昆虫与害虫防治}
% 学校所在地:一般不变,保持即可
\city{中国·广州}
% 授予学位时间:一般上半年是6月,下半年是12月,按照实际情况填写年月
\degreeConferralDate{2024}{6}

%- 学硕信息
\academic
% 指导老师
\advisor{陈某某~~教授}


% 是否专硕(不是专硕要取消下行代码,也即注释掉,专硕的信息会忽略)
\professional
% 第一指导教师
\advisorFirst{张某某\quad 教授}
% 第二指导教师
\advisorSecond{李~~四\quad 研究员}
% 专业学位类别
\professionalDegreeCategory{农业硕士}
% 领域
\field{农村发展}


%-
%-> 英文封面信息
%-
% TODO
% 英文标题:按照实际情况填写
\enTitle{Taxonomic Studies on the Subfamily Drosophila \\ (Diptera: Drosophilidae)}


%-
%-> 书脊信息
%-
% 论文中文题目
\spineTitle{仲\par 恺\par 农\par 业\par 工\par 程\par 学\par 院\par 学\par 位\par 论\par 文\par \rotatebox[origin=c]{-90}{~\LaTeX{}~}\par 模\par 板\par }
% 论文作者
\spineAuthor{厉\par 飞\par 雨}
% 学校名称
\spineInstitute{仲\\恺\\农\\业\\工\\程\\学\\院}
\spineYear{二\\${\bigcirc}$\\二\\四\\年}
%---------------------------------------------------------------------------%
