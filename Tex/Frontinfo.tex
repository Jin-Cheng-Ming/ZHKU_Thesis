%---------------------------------------------------------------------------%
%->> 封面信息 Titlepage information
%---------------------------------------------------------------------------%

%-
%-> 中文封面信息
%-

%- 通用信息
% 学校编码:一般不变,保持即可
\schoolCode{11347}
% 学号:按照自己的学号填写
\studentCode{2112907013}
% 按照实际情况填写
\classificationCode{S476}	
% 密级:按照实际情况填写,公开则不填
\confidential{} 
% 校徽与校名:一般不变,保持即可
\schoolLogo[height=2.2cm]{zhku_logo}
%- 标题:中括号(可选,默认封面标题中获取,即大括号内容)内容是给书签、文件名等预留的文本(一般一行字),
%       大括号(必填)内容是给封面标题排版的(可自定义换行),按实际情况填写
% 中文标题
% \title{东洋区冠果蝇亚科系统分类学研究}
\title[东洋区冠果蝇亚科系统分类学研究(双翅目:果蝇科)]{东洋区冠果蝇亚科系统分类学研究\\(双翅目:果蝇科)}
% 英文标题
% \enTitle{Taxonomic Studies on the Subfamily Drosophila(Diptera: Drosophilidae)}
\enTitle[Taxonomic Studies on the Subfamily Drosophila(Diptera: Drosophilidae)]{Taxonomic Studies on the Subfamily Drosophila\\(Diptera: Drosophilidae)}
% 作者,~~表示输出空格
\author{高~~某~~某}

% 培养单位:自己所在的学院或者研究所等,按照实际情况填写
\institute{农业与生物学院}
% 专业:按照实际情况填写
\major{农业昆虫与害虫防治}
% 学校所在地:一般不变,保持即可
\city{中国·广州}
% 授予学位时间:一般上半年是6月,下半年是12月,按照实际情况填写年月
\degreeConferralDate{2025}{6}

%- 学硕信息
% 指导老师:按照实际情况填写(~~表示输出空格)
\advisor{陈某某~~教授}

% 专硕信息
% 第一指导教师:按照实际情况填写(\quad 表示输出空格,注意代码与后续内容用空格隔开)
\advisorFirst{张某某\quad 教授}
% 第二指导教师:按照实际情况填写(\quad 表示输出空格,注意代码与后续内容用空格隔开)
\advisorSecond{李~~四\quad 研究员}
% 专业学位类别:按照实际情况填写
\professionalDegreeCategory{农业硕士}
% 领域:按照实际情况填写
\field{农村发展}


%-
%-> 英文封面信息
%-
% 英文标题:按照实际情况填写
\enTitle{Taxonomic Studies on the Subfamily Drosophila \\ (Diptera: Drosophilidae)}


%-
%-> 书脊信息
%-
% 论文中文题目:按照实际情况填写(注意要与封面页的标题一致,\\表示换行)
\spineTitle{仲\\恺\\农\\业\\工\\程\\学\\院\\学\\位\\论\\文\\\rotatebox[origin=c]{-90}{~\LaTeX{}~}\\模\\板\\}
% 论文作者:按照实际情况填写(注意要与封面页的作者一致,\\表示换行)
\spineAuthor{厉\\飞\\雨}
% 学校名称:一般不变,保持即可(\\表示换行)
\spineInstitute{仲\\恺\\农\\业\\工\\程\\学\\院}
% 论文发表年份:根据实际情况填写(注意要与封面页的授予学位时间一致,\\表示换行)
\spineYear{二\\${\bigcirc}$\\二\\五\\年}
%---------------------------------------------------------------------------%
