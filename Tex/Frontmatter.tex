%---------------------------------------------------------------------------%
%->> 前言部分事项 Frontmatter
%---------------------------------------------------------------------------%
%-
%-> 生成封面
%-
\intobmk*{\cleardoublepage}{\coverNameMark~\titleMark}
% 生成中文封面
\makeTitle
% TODO 生成英文封面
% \MAKETITLE

%-
%-> 作者声明
%-
% 生成声明页
\makeDeclaration

%-
%-> 中文摘要
%-
% 显示在书签但不显示在目录
\intobmk*{\cleardoublepage}{\abstractNameMark}
\chapter*{\abstractName}

中文摘要、英文摘要、目录、论文正文、参考文献、附录、致谢、攻读学位期间发表的学术论文与其他相关学术成果等均须由另页右页(奇数页)开始。

摘要是论文的缩影,语言力求精炼准确。应概括论文的主要信息,包括研究目的、内容、结果和结论,要重点突出论文的新见解或创新性。硕士论文摘要字数800字左右。

关键词采用能覆盖论文主要内容的通用技术词条作为关键词,一般3至5个,按涉及的内容、领域从大到小排在摘要下方。

英文摘要内容与中文摘要基本一致,写作力求符合科技英语文法要求,英文摘要由论文题目、作者、作者单位、正文、关键词组成。


\keywords{仲恺农业工程学院;学位论文;Latex模板}% 中文关键词

%-
%-> 英文摘要
%-
\intobmk*{\cleardoublepage}{\enAbstractNameMark}
\chapter*{\zihao{-3}\enTitleValue}

{
    \vspace{\baselineskip}
    \noindent\bfseries{\enAbstractName}
}

Chinese abstracts, English abstracts, table of contents, the main contents, references, appendix, acknowledgments, author's resume and academic papers published during the degree study and other relevant academic achievements must start with another right page (odd-numbered page).

\KEYWORDS{Zhongkai University of Agriculture and Engineering; Thesis; LaTeX Template}% 英文关键词
%---------------------------------------------------------------------------%
